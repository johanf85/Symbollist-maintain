\newglossaryentry{abszero}{name={Absolute zero},description={A theoretical condition concerning a system at zero kelvin, where a system does not emit or absorb energy (all atoms are at rest).}}
\newglossaryentry{base}{name={Base},description={A substance that accepts a proton and has a high pH; a common example is sodium hydroxide (NaOH).}}
\newglossaryentry{catalyst}{name={Catalyst},description={Any element or compound that facilitates an increase in the speed of a chemical reaction but which is not consumed or destroyed during the reaction. It is considered both a reactant and a product of the reaction.}}
\newglossaryentry{dipole}{name={Dipole},description={The electric or magnetic separation of charge.}}
\newglossaryentry{electshell}{name={Electron shell},description={An orbital around an atom's nucleus containing a fixed number of electrons (usually two or eight).}}
\newglossaryentry{frequency}{name={Frequency},description={A measurement of the number of cycles of a given process per unit of time. The SI unit for measuring frequency is the hertz (Hz), with 1 Hz = 1 cycle per second.}}
\newglossaryentry{gibbsenergy}{name={Gibbs energy},description={A value that indicates the spontaneity of a reaction. Usually symbolized as G.}}
\newglossaryentry{halogen}{name={Halogen},description={Any of the five non-metallic elements of Group 17 of the periodic table: fluorine (F), chlorine (Cl), bromine (Br), iodine (I), and astatine (At).}}
\newglossaryentry{igc}{name={ideal gas constant},description={The proportionality constant in the ideal gas law, defined as 0.08206 L·atm/(K·mol). Also called the universal gas constant.}}
\newglossaryentry{joule}{name={joule},description={The SI unit of energy (symbol: J); 1 joule = 1 Newton-meter.}}
\newglossaryentry{ketone}{name={ketone},description={An organic compound with a carbonyl group between two carbon atoms.}}
\newglossaryentry{liquid}{name={liquid},description={One of the four fundamental states of matter, characterized by nearly incompressible fluid particles that retain a definite volume but no fixed shape.}}
\newglossaryentry{mass}{name={mass},description={A property of physical matter that is a measure of its resistance to acceleration when a net force is applied. The SI base unit for mass is the kilogram (kg).}}
\newglossaryentry{neutron}{name={neutron},description={A neutral unit or subatomic particle that has no net charge.}}
\newglossaryentry{octetrule}{name={octet rule},description={A classical rule for describing the electron configuration of atoms in certain molecules: the maximum number of electron pairs that can be accommodated in the valence shell of an element in the first row of the periodic table is four (or eight total electrons). For elements in the second and subsequent rows, there are many exceptions to this rule. Also called the Lewis octet rule.}}
\newglossaryentry{ph}{name={pH},description={A measure of acidity or basicity of a solution.}}
\newglossaryentry{quantum}{name={quantum},description={The minimum amount of bundle of energy. Plural quanta.}}
\newglossaryentry{reactionrate}{name={Reaction rate},description={The speed at which reactants are converted into products in a chemical reaction.}}
\newglossaryentry{salt}{name={Salt},description={Any ionic compound composed of one or more anions and one or more cations.}}
\newglossaryentry{temperature}{name={Temperature},description={A proportional measure of the average kinetic energy of the random motions of the constituent microscopic particles of a system. The SI base unit for temperature is the kelvin.}}
\newglossaryentry{uamu}{name={Unified atomic mass unit},description={A unit of mass approximately equal to the mass of one proton or neutron and denoted with the symbol u; also called a Dalton and denoted with the symbol Da. Sometimes equated with the technically distinct and obsolete atomic mass unit and abbreviated amu.}}
\newglossaryentry{volume}{name={volume},description={The quantity of three-dimensional space enclosed by a closed surface, or the space that a substance (solid, liquid, gas, or plasma) or shape occupies or contains. The SI unit for volume is the cubic metre (m3).}}
\newglossaryentry{wavefunction}{name={Wave function},description={a function describing the electron's position in a three-dimensional space.}}
\newglossaryentry{xraydiffraction}{name={X-ray diffraction},description={a method for establishing structures of crystalline solids using singe wavelength X-rays and looking at diffraction pattern.}}
\newglossaryentry{yield}{name={yield},description={The quantifiable amount of product produced during a chemical reaction.}}
\newglossaryentry{zwitterion}{name={zwitterion},description={A chemical compound whose net charge is zero and hence is electrically neutral. But there are some positive and negative charges in it, due to the formal charge, owing to the partial charges of its constituent atoms.}}
